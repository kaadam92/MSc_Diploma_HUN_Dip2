%----------------------------------------------------------------------------
\chapter{Hyundai Technologies Center Hungary Kft.}
%----------------------------------------------------------------------------

\paragraph{}
A Hyundai Technologies Center Hungary a Hyundai Heavy Industries-nek egy magyarországi fejlesztő központja. A cég 18 éve képviselteti magát a magyar műszaki életben. Bár a névről az első gondolat az autógyártó lehet, azonban a Hyundai egy ennél sokkal nagyobb cég, az autógyár csak egy szelete. A H-TEC a Hyundai Heavy Industries-nek a leányvállalata. A mi cégünk így a HHI-tól kapja a kutatás-fejlesztési megrendeléseket, szorosan együttműködve a dél-koreai kollégákkal.

\begin{figure}[h]
	\centering
	\includegraphics[width = 0.8\textwidth]{figures/hyundai_logo.png}
	\caption{Hyundai HHI} 
	\label{fig:hhi}
\end{figure}

\paragraph{}
Az itthoni tevékenysége a cégnek 4 nagyobb csoporra osztható. Ez a beosztás a következőképpen alakul:


\begin{itemize}
	\item{GIS - Gas Insulated Switchgear}
	\item{RM - Rotary Machines}
	\item{TM - Trasformer Machines}
	\item{PE - Power Electronics}
\end{itemize}


\paragraph{}
A részleg, melyben dolgozom a \emph{Power Electronics}\footnote{PE Team} a teljesítményelekotrnikai fejlesztő csoport, a dolgozat írásánk idején ünnepli tizedik évfordulóját hazánkban. Pályafutása során a csoport először szélerűművek teljesítméynelektornikai támogaásához fejlesztett konvertereket. Ezt követően is a megújuló energiaforrások felé irányult a fókusz, naperőművekhez fejlesztett invertereket a háztartásitól az ipari méretig. A csapatban körülbelül huszonöten dolgoznak, a munka az alábbi csoportok között oszlik meg:

\begin{itemize}
	\item{Power Conversion}
	\item{Motor Control}
	\item{Master Control}
	\item{Embedded Control}
	\item{Mechanical}
	\item{Laboratory} 
\end{itemize}

Szerencsére kipróbálhattam magam sok csoportban, foglalkoztam szoftver teszttel, vezélrő algoritmusokkal, illetve a beágyazott szoftverrendszer fejlesztésével is. 



