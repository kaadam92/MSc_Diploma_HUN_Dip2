%----------------------------------------------------------------------------
\chapter{Hyundai Technologies Center Hungary Kft.}
%----------------------------------------------------------------------------

\begin{figure}[h]
	\centering
	\includegraphics[width = 0.8\textwidth]{figures/hyundai_logo.png}
	\caption{Hyundai HHI} 
	\label{fig:hhi}
\end{figure}

\paragraph{}
A Hyundai Technologies Center Hungary a Hyundai Heavy Industries-nek egy magyarországi fejlesztő központja. A cég 18 éve képviselteti magát a magyar műszaki életben. Bár hazánkban a cég leginkább az autómárka révén ismert, ez csak egy szelete a vállalat tevékenységének. Ezen felül foglalkozik még a háztartási gépek gyártásatól kezdve, az ipari és munkagépekent át egészen a hajógysártásig szinte mindennel, sőt a Hyundai Heavy Industries a világ legnagyobb hajógyára.  A H-TEC a HHI-nek a leányvállalata. A mi cégünk így a HHI-tól kapja a kutatás-fejlesztési megrendeléseket, szorosan együttműködve a dél-koreai kollégákkal.

\paragraph{}
Az itthoni tevékenysége a cégnek 4 nagyobb csoporra osztható. Ez a beosztás a következőképpen alakul:


\begin{itemize}
	\item{GIS - Gas Insulated Switchgear - Gáz szigetelésű kapcsolószerkezetek}
	\item{RM - Rotary Machines - Villamos forgógépek}
	\item{TM - Trasformer Machines - Transzformátorok}
	\item{PE - Power Electronics - Teljesítményelektronika}
\end{itemize}




\paragraph{}
A részleg, melyben dolgozom a \emph{Power Electronics}\footnote{PE Team} a teljesítményelekotrnikai fejlesztő csoport, a dolgozat írásánk idején ünnepli tizedik évfordulóját hazánkban. Pályafutása során a csoport először szélerűművek teljesítméynelektornikai támogaásához fejlesztett konvertereket. Ezt követően is a megújuló energiaforrások felé irányult a fókusz, naperőművekhez fejlesztett invertereket a háztartásitól az ipari méretig. A csapatban körülbelül huszonöten dolgoznak, a munka az alábbi csoportok között oszlik meg:

\begin{itemize}
	\item{Power Conversion}
	\item{Motor Control}
	\item{Master Control}
	\item{Embedded Control}
	\item{Mechanical}
	\item{Laboratory} 
\end{itemize}

Szerencsére kipróbálhattam magam sok csoportban, foglalkoztam szoftver teszttel, vezélrő algoritmusokkal, illetve a beágyazott szoftverrendszer fejlesztésével is. A dolgozat írása alatt az \emph{Embedded Control} és a \emph{Motor Control} csoportoknak képeztem szerves részét, ugyanis ezen csoportok munkájához van a legnagyobb szükség a Hardware-in-the-Loop szimulátorra. Természetesen a csoportnak a korábbi projektek révén már ebben a témában is jelentős tapasztalata van, így rengeteget tudtam tanulni kollegáim korábbi munkájából, illetve nagyon jó kiindulási alapot biztosítottak az eddig elért eredmények.

\improvement[inline]{Itt még lehetne dolgozni, nem tudom elég-e.}



