\chapter{Piackutatás}
\section{Opal-RT \cite{opal}}

\paragraph{}
Az Opal-RT egy francia cég, 1997-ben alapította Jean Bélanger és Lise Laforce. Céljuk, hogy a Hadware-in-the-Loop szimulációt a mérnökök munkájának szerves részévé tegyék, megkönnyítve ezzel a munkájukat. Az iparban számod területen megjelentek már, az űrkutatásban, az repülésben, az autó iparban és a teljesítményelektronikában is. Ezen felül nagy hangsúlyt fektetnek az egyetemi oktatásra, és a felső oktatási intézményekkel való együttműködésre.

\paragraph{}
Termék palettájuk leginkább szoftverben változatos. A hardvert tekintve moduláris felépítésű eszközöket lehet tőlük vásárolni, melyek így tetszőlegesen bővíthetőek megfelelő számú és típusú analóg vagy digitális be és kimenettel. Hardveres megoldásaik FPGA-ra alapulnak, a nagy és specifikus számítás igény kielégítésének érdekében.

\paragraph{}
Szoftvereik az RT-LAB szoftver csomag köré épülnek, mely teljesen integrált a Simulinkkel. A szoftver rugalmassága és skálázhatósága megengedi ,hogy szinte bármilyen szimulációs feladatot el lehessen vele végezni. Ezt egészítik ki még egyék komponensek, a HYPERSIM, az eMEGAsim és az ePHASORsim. Ezek a szoftverek támogatják nagy energia rendszerek, például egy gyár vagy egy erőmű villamos hálózatának a szimulációját is. Kifejezettem teljesítményelektronikai feladatokra tervezték az eDRIVEsim, az eFPGAsim, az FPGA eHS és RT-XSG szoftvereket. Ezek alkalmasak bonyolultabb teljesítmény elektronikai struktúrák szimulációjára, illetve a különböző védelmek és vezérlések tesztelésére is.

\section{dSPACE \cite{dspace}}

A dSPACE egy 1988-ban alapított német cég. Profiljukat tekintve szintén HIL szimulátorok fejlesztésével foglalkoznak, bár fő profiljuk inkább az autóipar. Ebben a szegmensben a HIL szimulációnak rendszer integrációs szerepe van.

\paragraph{}
Egy mai modern járműben számos különálló elektronikus vezérlő egység (ECU) található, melyek valamilyen buszon kommunikálnak egymással, többnyire CAN buszon. Emiatt a tesztelést és a fejlesztést nehézkessé teszi ,hogy valós működés szimulációjához, ténylegesen szükség van a más szenzoroktól és eszközöktől érkező adatra, még jobb, ha ezek az adatok hasonlítanak egy valós jármű által produkált értékekre. Ennek a mélységét jól mutatja ,hogy a dSPACE eszközei még az autó akkumulátorát is képesek szimulálni, így ellenőrizhetővé téve például egy akkumulátor merülés, vagy egy motor indítás hatását a rendszerre.

\paragraph{}
A modern igényeket kielégítve ők is kínálnak szimulációs megoldásokat villamos hajtásokhoz, a problémát szintén autóipari szemszögből megközelítve. Termékeik alkalmasak mind a teljesítményelektronika szimulációjára, csak a logikai jeleket ellenőrizve. De alkalmasak akár úgynevezett P-HIL ként, csak a motort és a mechanikus terhelést szimulálva elnyelni az elkészült erősáramú elektronika által kiadott teljesítményt.