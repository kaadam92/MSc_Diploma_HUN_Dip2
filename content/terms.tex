%----------------------------------------------------------------------------
\chapter*{A dolgozatban alkalmazott rövidítések feloldása}\addcontentsline{toc}{chapter}{Rövidítések}
%----------------------------------------------------------------------------


\begin{description}
\item[AFE]	Active front end - aktív bemenet
\item[CCB]  Converter Control Board - konverter vezérlő kártya
\item[CSI]	Current source inverter - Áramforrásként tekinthető inverter
\item[CSR]	Current source rectifier - Áramforrásként tekinthető egyenirányító
\item[GCT]	Gate-controlled thyristor - Gate-vezérelt tirisztor
\item[GTO]	Gate turn-off thyristor - Gate-el kikapcsolható tirisztor
\item[HIL]  Haedware-in-the-Loop
\item[HMI]  Human Machine Interface
\item[IGBT]	Insulated gate bipolar transistor
\item[JTAG] Joint Test Action Group - debugger interfész
\item[LCI]	Load commutated inverter - Olyan inverter, melyben a kommutávió a terhelés hatására következik be
\item[LV]	Low voltage - kisfeszültség
\item[MV]	Medium voltage - középfeszültség
\item[MCB]  MAster Conrol Board - Fő vezérlő kártya
\item[PIB]  Power interface board - teljesítmény meghajtó kártya
\item[PM]	Permanent magnet - állandó mágnes
\item[PMSM]	Permanent magnet synchronous machine - állandó mágneses szinkrongép
\item[PWM]	Pulse-width modulation - impulzuszélesség moduláció
\item[TAB]  Test Adapter Board - Teszt kártya
\item[VSI]	Voltage source inverter - Feszültségforrásként tekinthető inverter
\item[VVI]	Variable-voltage inverter - Változtatható feszültségű inverter
\item[UART] Universal asynchronous receiver/transmitter - soros port
\end{description}
