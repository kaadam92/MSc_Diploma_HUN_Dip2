\chapter{Egy frekvenciaváltó tervezése}

\paragraph{}
A frekvenciaváltó egy olyan eszköz mely váltakozó áramú bemenetből váltakozó áramú kimenetet állít elő, mint ahogy a neve is mutatja, más frekvenciával vagy akár feszültséggel, mint a bemenet. Erre azért van szükség, mert a meghajtani kívánt folyamatnak nagyon valószínű, hogy más igényei vannak, mint amit a hálózat önmagában képes biztosítani. Értem ez alatt azt, hogy közvetlen összeköttetés esetén a motorunk $50\ Hz-el$, vagy ennek egész számú hányadosával tudna forogni, illetve ez a paraméter fizikailag csak a motor módosításával lehetséges. Az igény tehát nyilvánvaló az eszközre.

\paragraph{}
A korai megoldás

\section{Az eszköz felépítése}



\section{A szükséges kompetenciák}
\section{A fejlesztés folyamata}
