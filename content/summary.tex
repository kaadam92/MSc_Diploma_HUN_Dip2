\chapter{A Simulink modell felépítése}