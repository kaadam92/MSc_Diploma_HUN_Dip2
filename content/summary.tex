\chapter{Összegzés}

\paragraph{}
Munkám során volt szerencsém egy úttörő technológiát megismerni. A téma mélységét tekintve hatalmas, a fent leírtak csak a jéghegy csúcsát jelentik. Remélem a jövőben lesz lehetőségem tovább fejleszteni az elkészült eszközt, mert még nagyon sok lehetőség rejlik benne és nagyon sokat lehet tanulni a folyamatból.

Külön örömmel töltött el munkám során amikor kollégáim használni tudták a szimulátort, segítségül munkájukhoz. Egyértelműen látszik, hogy egy ilyen eszköz rendkívüli mértékben növeli a munka hatékonyságát. Az iparban nem sokan foglalkoznak Hardware-in-the-Loop szimulátorok fejlesztésével, a cégeket akiket a témában megismertem az Opal-RT és a dSPACE. Ők leginkább moduláris rendszert fejlesztenek, mely a felhasználói igény szerint szerelhető össze. Mint önálló termék, ez valóban életképesebb megoldásnak tűnik, azonban itt egy projekt specifikus cél eszközt fejlesztünk.

Az új hardver elkészülte után célunk több példányt készíteni az eszközből. A koreai kollégáknak is szükségük lesz rá, Magyarországon is több példányon zajlik párhuzamosan a fejlesztés. Ezen felül a különböző testesetek párhuzamos futtatásával a tesztelés folyamata is gyorsítható lesz.

Dolgozatom reményeim szerint egy átfogó képet adott a fejlesztés folyamatáról, a használat miben létéről. Ha sikerül a témában megfelelő mélységig hatolni, cikkek formájában folytatni szeretném a téma publikálását.
