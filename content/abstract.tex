\pagenumbering{roman}
\setcounter{page}{1}

\selecthungarian

%----------------------------------------------------------------------------
% Abstract in Hungarian
%----------------------------------------------------------------------------
\chapter*{Kivonat}\addcontentsline{toc}{chapter}{Kivonat}

A teljesítmény-elektronikai eszközök mindig is jelentős szerepet töltöttek be az iparban és mindennapjainkban. Gondoljunk csak a telefonunk töltőjére, vagy bármilyen háztartási eszközünk tápegységére. Ipari területen is szükség van hajtásvezérlő elemekre (pl. frekvenciaváltók), vagy nagy teljesítményű tápegységekre, különböző hálózatjavító eszközökre. Napjainkban egyre inkább elterjedőben vannak az elektromos hajtású járművek is, illetve a megújuló energiaforrásokból előállított energiát is a hálózatba kell juttatni, pl. szolár-inverterekkel.

A dolgozat az ilyen eszközök fejlesztésének egy módjáról, nevezetesen a Hardware in the Loop szimulációról értekezik. A módszer lehetőséget biztosít a mérnökök számára, hogy párhuzamosítsák a szofver és hardware tervezési feladatokat. Ehhez implementálni kell hozni a fizikai eszköz modelljét, majd ezt a modellt, a megfelelő szimulációs lépésköz elérése érdekében valamilyen céleszközön, többnyire FPGA-n futtatni. Szükség van továbbá arra, hogy a folyamatot meg is tudjuk figyelni, a folyamat paramétereit megfelelő felbontással ki tudjuk olvasni, hogy validálni tudjuk a működést. A külső hardware-en való futás további előnye, hogy akár a végleges elkészült vezérlő elektronikát is összeköthetjük a szimulációt végző FPGA-val. Így annélkül tudjuk tesztelni a kész terméket, hogy a valós főköri elemeket biztosan nem éri károsodás, illetve a valós teljesítmény sem jelenik meg, így egy esetleges fel nem tárt hibát anyagi és fizikai kockázatok nélkül tudunk felderíteni.

A tárgyalás kitér egy ilyen szimulációs környezet felépítésére, annak nehézségeire. Példaképp bemutat néhány alkalmazást, illetve rávilágít a módszer korlátaira is.


\vfill
\selectenglish


%----------------------------------------------------------------------------
% Abstract in English
%----------------------------------------------------------------------------
\chapter*{Abstract}\addcontentsline{toc}{chapter}{Abstract}

Power electronic devices have always played a huge role in industrial and everyday usage. Go no further than mobile phone chargers or power units of any household device to see the evidence of it. Today, even a single LED light bulb contains a small electronic device, wich generates the constant current supply from grid voltage.
 
In the industrial field, there is also an enormous demand for motor controllers, such as frequency converters or power units with huge output capacity, even in the megawatt range. Electric cars are also gaining market share as are renewable energy sources. In both cases, high-power converters are needed to charge the cars or generate alternating current from DC solar panels.
 
In my thesis, I will discuss the development of such devices, with emphasis on hardware-in-the-loop simulators. This method provides a solution to engineers who develop the software and the power electronics side by side. For this form of usage, we have to develop the mathematical model of the real hardware, then synthesize it on hardware with immense computing capability, usually on an FPGA. We must also implement monitoring capabilities in the system to validate the simulation or modify the parameters. The other advantage of physical hardware is that it can be connected to real control electronics, with real interfaces. In this way, we can test our control algorithms and methods without causing costly damage to the main power circuit. 
 
I will examine the simulation environment itself, expand on the development process, and consider systematic difficulties. I will provide an example solution and show both the advantages and the bottlenecks of the system.


\vfill
\selectthesislanguage

\newcounter{romanPage}
\setcounter{romanPage}{\value{page}}
\stepcounter{romanPage}