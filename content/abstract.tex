\pagenumbering{roman}
\setcounter{page}{1}

\selecthungarian

%----------------------------------------------------------------------------
% Abstract in Hungarian
%----------------------------------------------------------------------------
\chapter*{Kivonat}\addcontentsline{toc}{chapter}{Kivonat}

A teljesítmény-elektronikai eszközök mindig is jelentős szerepet töltöttek be az iparban és mindennapjainkban. Gondoljunk csak a telefonunk töltőjére, vagy bármilyen háztartási eszközünk tápegységére. Ipari területen is szükség van hajtásvezérlő elemekre (pl. frekvenciaváltók), vagy nagy teljesítményű tápegységekre, különböző hálózatjavító eszközökre. Napjainkban egyre inkább elterjedőben vannak az elektromos hajtású járművek is, illetve a megújuló energiaforrásokból előállított energiát is a hálózatba kell juttatni, pl. szolár-inverterekkel.

A dolgozat az ilyen eszközök fejlesztésének egy módjáról, nevezetesen a Hardware in the Loop szimulációról értekezik. A módszer lehetőséget biztosít a mérnökök számára, hogy párhuzamosítsák a szofver és hardware tervezési feladatokat. Ehhez implementálni kell hozni a fizikai eszköz modelljét, majd ezt a modellt, a megfelelő szimulációs lépésköz elérése érdekében valamilyen céleszközön, többnyire FPGA-n futtatni. Szükség van továbbá arra, hogy a folyamatot meg is tudjuk figyelni, a folyamat paramétereit megfelelő felbontással ki tudjuk olvasni, hogy validálni tudjuk a működést. A külső hardware-en való futás további előnye, hogy akár a végleges elkészült vezérlő elektronikát is összeköthetjük a szimulációt végző FPGA-val. Így annélkül tudjuk tesztelni a kész terméket, hogy a valós főköri elemeket biztosan nem éri károsodás, illetve a valós teljesítmény sem jelenik meg, így egy esetleges fel nem tárt hibát anyagi és fizikai kockázatok nélkül tudunk felderíteni.

A tárgyalás kitér egy ilyen szimulációs környezet felépítésére, annak nehézségeire. Példaképp bemutat néhány alkalmazást, illetve rávilágít a módszer korlátaira is.


\vfill
\selectenglish


%----------------------------------------------------------------------------
% Abstract in English
%----------------------------------------------------------------------------
\chapter*{Abstract}\addcontentsline{toc}{chapter}{Abstract}

Power Electronic devices have always been played a huge roll in the industial and the everyday usage. Go no further then the charger of our mobile devices or the power unit of every household device, today even in a single modern LED lightbulb contains a small electronic device, wich generates the constant current supply from the grid voltage. In the industrial field, there is also a huge demand of motor controllers, such as frequency converters, or big power units, even int the $MW$ range. Nowadays electic cars are also gaining more marketshare as well as the renewable energy sources. In both cases, high power power conerters needed to charge the car or generate alternating current from the DC solar panels.

In my thesis I will discuss the development process of such kind of devices, with a great emphasis on the Hardware-in-the-Loop simulators. This method provides a solution to the engineers to develop the software and the power electronics parallel. For this kind of usage, we have to develop the mathematical modell of the real hardware, then we need to synthetise it on a hardware with huge coputing capabilty, usually on a FPGA. There is also a need to monitorig capabilities of the system, to validate the simulation or modify the parameters. The other advantage of the pysical hardware, it can be connected with the real control electronics, with the real interface. In this way we can tst our control algorythms and methods without cousing any costly damage in the main power circuit. 

I will write about the simulation environment itself, about the development process of it, and about the dificulties as well. I will provide an example solution, and show tha advantages the bottlenecks of the system.


\vfill
\selectthesislanguage

\newcounter{romanPage}
\setcounter{romanPage}{\value{page}}
\stepcounter{romanPage}